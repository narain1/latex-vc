\documentclass[12pt]{article}

\usepackage{amsmath}
\usepackage{graphicx}
\usepackage{hyperref}

\title{Sample \LaTeX\ Document}
\author{Author Name}
\date{\today}

\begin{document}

\maketitle

\section{Introduction}
This is a simple LaTeX document to show some basic features. For more information, see the \href{https://www.latex-project.org/}{LaTeX project website}.

\section{Mathematics}
\LaTeX\ is great for typesetting mathematics. For example, the quadratic formula looks like this in \LaTeX:
\[
x = \frac{{-b \pm \sqrt{{b^2 - 4ac}}}}{{2a}}
\]
You can also align equations:
\begin{align*}
a & = b + c \\
x & = y - z
\end{align*}

\section{Tables}
Tables can be created as follows:
\begin{table}[h!]
\centering
\begin{tabular}{|c|c|c|}
\hline
\textbf{Column 1} & \textbf{Column 2} & \textbf{Column 3} \\
\hline
Row 1 & Data 1 & Data 2 \\
Row 2 & Data 3 & Data 4 \\
\hline
\end{tabular}
\caption{Example of a table}
\label{tab:example_table}
\end{table}

\section{Figures}
You can include images like this:
\begin{figure}[h!]
\centering
  \includegraphics[width=0.5\textwidth]{images/sample.jpg}
\caption{An example image.}
\label{fig:example_figure}
\end{figure}

\section{Conclusion}
Thank you for exploring this sample \LaTeX\ document. Remember, the more you practice, the better you'll get!

\end{document}

